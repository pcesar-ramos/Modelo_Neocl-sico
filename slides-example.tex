% --------------------------------------------------------------------
% LaTeX beamer slide example for Leiden slides
% Copyright LaTeX template: 2009-2010 by Joost Schalken
% --------------------------------------------------------------------

% ====================================================================
% Setup of slide environment
\batchmode % Do not display issues with package loading
\documentclass[t,11pt]{beamer}

% --------------------------------------------------------------------
% Load packages
\usepackage{booktabs}
\usepackage{graphicx}
\usepackage{tikz}
\usetikzlibrary{calc,trees,positioning,arrows,chains,shapes.geometric,%
    decorations.pathreplacing,decorations.pathmorphing,shapes,%
    matrix,shapes.symbols}
\usepackage{listings}
\lstset{tabsize=2,showspaces=false,showtabs=false,basicstyle=\ttfamily\mdseries\itshape\normalsize}

% --------------------------------------------------------------------
% Beamer version theme settings
\usetheme[
    faculty=sciences,  % humanities, law, medicine, sciences, socialsciences
    lang=en,           % en, nl
    rmfont=pmn,
    logofont=fpi,
    %totalpages=off    % Disable total number of slides
]{leiden}
%\usetheme[faculty=sciences,lang=en]{leiden}

% Remove navigation symbols
\setbeamertemplate{navigation symbols}{}

% Better default font
\usepackage{iwona}
\usepackage[textfont={scriptsize,it}]{caption}
\setbeamerfont{caption}{size=\scriptsize}
\renewcommand*{\familydefault}{\sfdefault}




% --------------------------------------------------------------------
\def\liketitle#1{%
{\usebeamerfont{frametitle}\usebeamercolor[fg]{frametitle}%
\begin{flushleft}%
\vspace{-\baselineskip}% Cometic correction for space introduced by flushleft
#1\par
\end{flushleft}%
\vspace{-\baselineskip}% Cosmetic correction for space introduced by flushleft
}%
\vspace{0.75\baselineskip}%
}

% --------------------------------------------------------------------
\setbeameroption{hide notes}
%\setbeameroption{show notes}
%\setbeameroption{show notes on second screen}

% --------------------------------------------------------------------
\nonstopmode % Include issues with the slides


% ====================================================================
% Header settings
\def\lecturename{Lecture}
\lecture[Leiden Template]{Macroeconomía}{ldn-bmr}
\subtitle{Crecimiento Económico}
\date{December 9th, 2010}
\title{\insertlecture}
\author{Macroeconomía}
\institute{2020}
\subject{Lecture: \lecturename}


% ====================================================================
% Main part

% --------------------------------------------------------------------
\batchmode % Do not include issues with the package definitions
\begin{document}
\nonstopmode % Include issues with the slides


% ====================================================================
\section*{Introduction}

% --------------------------------------------------------------------

\setbeamertemplate{navigation symbols}{}
\begin{frame}[plain]
  \maketitle
\end{frame}
\addtocounter{framenumber}{-1}% don't count the title slide.


\begin{frame}{Introducción}
Teoría del Crecimiento: Modelo Neoclásico. 
Modelo de Sollow \& Swan (1956)
\begin{itemize}
    \item Fase de crecimiento - estructura de la sociedad.
    \item Factores que explican el crecimiento de largo plazo.
\end{itemize}
La modelizacion del crecimiento inicia a finales de la decada de 1930. Los primeros trabajos: Harrod (1939) y Domar (1946). 
\medskip 

En 1956 Solow \& Swan elaboraron el modelo basico de una economia dinamica (funcion de produccion Neoclasica con Rendimientos Constantes a escala R.C.E. Y Rendimientos decreciente de los factores). 
\end{frame}

%--------------------------------------------%
%--------------------------------------------%
\section{}
%--------------------------------------------%
%--------------------------------------------%

 \begin{frame}{Introduccion}
 En 1965 Cass \& Koopmans a partir del modelo elaborado por Ramsey (1928) completaron la teoria Neoclasica con el supuesto de que los consumidores son optimizadores racionales.
 \medskip
 
 En la segunda mitad de los ochenta, con los trabajos pioners de Romer (1986), Lucas (1988) \& Rebelo (1991). Introdujeron del progreso tecnologico
 
 \end{frame}
%--------------------------------------------%
%--------------------------------------------%
\begin{frame}{Sollow 1956}
\begin{itemize}
    \item Equilibrio ahorro - inversion
    \item Vaciado de mercados
    \item funcion de produccion lineal y homogenea en logaritmos
    \item Sustituibilidad de capital y trabajo
    \item Relacion ahorro/renta es constante
\end{itemize}
    
\end{frame}
%--------------------------------------------%
%--------------------------------------------%

%============================================%
%============================================%
\section{Ejemplos}

%--------------------------------------------%
%--------------------------------------------%
\begin{frame}{Ejemplo}
Dada una ecuación lineal aplicamos el algoritmo de Newton y el valor inicial es $x=4$, entonces:
$$F(x)=x-2$$
La solución que satisface $F(x)=0$ es para $x=2$ en este ejemplo apliquemos el algortmo se tiene:
$$x_1 = x_0 - \frac{F(x_0)}{F'(x)}=4-\frac{4-2}{1}=2$$
Para comprobar $F(x_1)=0$ reemplazamos la solucion $x_1$:
$$F(x_1)=0 \rightarrow F(x_1)=2-2=0$$
\end{frame}
%--------------------------------------------%
%--------------------------------------------%
\begin{frame}{Frame Title}
    
\end{frame}
%--------------------------------------------%
%--------------------------------------------%
 
% ====================================================================
\section{Final}


% --------------------------------------------------------------------
\begin{frame}{Final presentación}
\vfill % Vertical centering
\begin{center}
\alert{\large Crecimiento Económico}\\
{\LARGE Macroeconomía}\\
{\tiny 2020}
\end{center}
\end{frame}

\end{document}
